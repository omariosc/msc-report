\section{Introduction}

\subsection{Minimally Invasive Surgery}

Minimally Invasive Surgery (MIS) represents a significant advancement in surgical procedures by minimising tissue damage and reducing complications \cite{jaffray_minimally_2005}. Laparoscopy, a subset of MIS, involves the insertion of a laparoscope — a long, thin tube with high-intensity light and a high-resolution camera at the front — into the abdomen through a small incision, allowing surgeons to view and operate on the internal organs more precisely \cite{monnet_laparoscopy_2003}. Laparoscopy is an attractive option, particularly in resource-constrained environments (RCEs) where hospital stays and postoperative care can be expensive and logistically challenging \cite{rockall_laparoscopy_2014}. However, adopting MIS is often hindered by the scarcity of skilled surgeons, the high cost of laparoscopic equipment \cite{meara_global_2015}, and the surgeon must deal with difficult hand-eye coordination, restricted mobility and a narrow field of view, inevitably resulting in poorer quality data \cite{bodenstedt_comparative_2018}. Despite these challenges, laparoscopy remains the preferred method for many surgeries due to its benefits, such as reduced blood loss, lower infection rates, and faster recovery times than open surgery \cite{jaffray_minimally_2005}.

\subsubsection{Low and Middle-Income Countries}

In 2015, the Lancet Commission on Global Surgery emphasised the urgent need for increased volume and quality of surgery as an essential part of global health \cite{meara_global_2015}. Significant disparities in access to surgical care in LMICs contribute to an estimated 5 billion people lacking safe and affordable surgical services, with an additional 143 million surgeries annually needed to fill this gap. Furthermore, the World Health Organization has highlighted a global shortfall of 44.5 million health workers, with a considerable proportion of this deficit being in surgery \cite{world_health_organization_world_2016}. This shortage of skilled surgeons in LMICs is a substantial barrier to improving surgical outcomes and reducing the burden of surgical diseases in these regions. 

%%The Lancet Commission on Global Surgery (LCoGS) 2015 emphasised the urgent need for increased volume and quality of surgery as an essential part of global health \cite{meara_global_2015}. It recognised that developing safe, essential, life-saving surgical care in low- and middle-income countries (LMICs) has lagged behind developed nations and requires significant expansion. Significant disparities in access to surgical care compared to high-income countries (HICs) contribute to an estimated 5 billion people lacking safe and affordable surgical services, with an additional 143 million surgeries annually needed to fill this gap. 11\% of deaths in LMICs are due to conditions treatable by surgery, with 80\% of them being preventable. Furthermore, the World Health Organization (WHO) has highlighted a global shortfall of 44.5 million health workers, with a considerable proportion of this deficit being in surgery and anaesthesia \cite{world_health_organization_world_2016}. This shortage of skilled surgeons in LMICs is a substantial barrier to improving surgical outcomes and reducing the burden of surgical diseases in these regions. 

\subsection{Surgical Skill Assessment}

The global shortage of skilled surgeons significantly impacts patient outcomes, with wide variability in surgical skills leading to complications and avoidable harm \cite{jin_tool_2018}. Assessing operative skills is essential to improving surgical training. Traditional skill assessment methods are time-consuming, subjective, often rely on expert surgeons to evaluate the performance of an entire operation manually \cite{vassiliou_global_2005, paley_crowdsourced_2021} and are prone to human bias \cite{levin_automated_2019}. There is a growing emphasis on developing automated real-time surgical tool detection and tracking systems to address these limitations, offering objective and consistent evaluations \cite{loza_realtime_2024}. Various computer vision applications for tooltip detection and tracking have been developed for laparoscopic surgery training \cite{matsumoto_laparoscopic_2022}. Detection, tracking, localisation, and pose estimation techniques using computer vision and Artificial Intelligence (AI) could provide insight into surgical performance \cite{bodenstedt_comparative_2018, allan_toward_2013, constable_enhancing_2024}. Tooltip motion primitives may directly correlate to improving skill \cite{retrosi_motion_2015}. This establishes the motivation to develop models which accurately detect and track surgical tools in laparoscopic videos for use in the computer-assisted intervention of skill assessment in LMICs \cite{nwoye_cholectrack20_2023}. 
% These tools have the potential to provide feedback to trainee surgeons, investigate the aspects of skill linked to patient outcomes, and assist educators in determining if trainees meet competency thresholds.

\subsection{Objectives}

The primary objective of this study is to detect and track surgical tools using various SOTA models on the in-house Artificial Intelligence Enhanced Laparoscopic Training (AI-ELT) dataset to aid future research in in-vitro laparoscopic datasets in RCEs.

\subsection{Contributions}

We can summarise our contributions as follows:

\begin{itemize}[noitemsep, left=0pt]
\item We introduce a novel dataset of laparoscopic training videos, AI-ELT, the first of its kind for surgical skill analysis in non-in-vivo contexts with abundant high-quality annotations.
\item We develop and compare SOTA deep learning models for tool detection in laparoscopic videos, focusing on anchor-based and anchor-free architectures, inference times and model complexity.
% \item We develop and compare SOTA deep learning models for surgical tool detection in laparoscopic videos, focusing on anchor-based and anchor-free architectures, inference times and model complexity, providing valuable insights for future research in surgical training in RCEs.
\item We develop a surgical tool tracking algorithm for the AI-ELT dataset, achieving 100\% accuracy over already detected tools and tooltips, setting a foundation for real-time feedback and skill assessment in surgical training programs.
\end{itemize}

% TABLE OF ACRONYMS
% \begin{table}[h]
%     \centering
%     \caption{Table of Acronyms}
%     \begin{tabular}{ll}
%       \toprule
%       \textbf{Acronym} & \textbf{Definition} \\
%       \midrule
%       6DoF & Six Degrees-of-Freedom \\
%       AI & Artificial Intelligence \\
%       AI-ELT & AI Enhanced Laparoscopic Training \\
%       COCO & Common Objects in Context \\
%       FPS & Frames Per Second \\
%       HICs & High-Income Countries \\
%       IoU & Intersection over Union \\
%       LCoGS & Lancet Commission on Global Surgery \\
%       LMICs & Low and Middle-Income Countries \\
%       MIS & Minimally Invasive Surgery \\
%       mAP & Mean Average Precision \\
%       SOTA & State-of-the-Art \\
%       WHO & World Health Organization \\
%       \bottomrule
%     \end{tabular}
%     \label{tab:acronyms}
%   \end{table}
